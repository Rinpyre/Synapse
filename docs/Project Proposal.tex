\documentclass[12pt, a4paper]{article}

\usepackage{graphicx}
\usepackage[utf8]{inputenc}
\usepackage{geometry}
\usepackage[hidelinks]{hyperref}
\usepackage{titlesec}
\usepackage{fancyhdr}
\usepackage{tocloft}
\usepackage{amssymb}
\usepackage{float}
\usepackage{color}

\geometry{margin=1in}
\setlength{\headheight}{17.2pt}

% Header and footer setup
\pagestyle{fancy}
\fancyhf{}
\fancyhead[L]{Synapse - Group 12}
\fancyfoot[C]{\thepage}

% Section title formatting - reduce spacing and add periods
\titleformat{\section}
  {\large\bfseries}
  {\thesection.}
  {0.25em}
  {}

% Subsection title formatting
\titleformat{\subsection}
  {\normalsize\bfseries}
  {\thesubsection.}
  {0.25em}
  {}

% TOC customization - add dots to all entries and center title
\renewcommand{\cftsecleader}{\cftdotfill{\cftdotsep}}
\renewcommand{\cftsubsecleader}{\cftdotfill{\cftdotsep}}
\renewcommand{\cftsecaftersnum}{.}
\renewcommand{\cftsubsecaftersnum}{.}
\renewcommand{\cftsecnumwidth}{1.5em}
\renewcommand{\cftsubsecnumwidth}{2em}

% Center TOC title
\renewcommand{\contentsname}{\centering\large\bfseries Table of Contents}
\renewcommand{\cfttoctitlefont}{\hfill}
\renewcommand{\cftaftertoctitle}{\hfill}

% 2. DOCUMENT CONTENT
\begin{document}

% Custom title page
\begin{titlepage}
    \centering
    \vspace*{2cm}

    {\Huge\bfseries Synapse \par}
    \vfill
    {\Large\bfseries Project Proposal \par}
    {\large\bfseries AI Summarisation and Analysis of Resource \par}
    \vspace{1.5cm}

    \vfill
    {\Large \textbf{Group 12}} \\
    \vspace{0.25cm}
    \begin{tabular}{r | l}
        Adrian Cristian Stancu & astan24@student.sdu.dk \\
        Dorina Petra Nagy      & donag24@student.sdu.dk \\
        Tomass Zarins          & tozar24@student.sdu.dk \\
        Jakub Cuninka          & jacun24@student.sdu.dk \\
        Marks Karlins          & mkarl24@student.sdu.dk \\
        Balazs Istvan Nemeth   & banem24@student.sdu.dk \\
    \end{tabular}
    \vspace{1cm}

    {\Large \textbf{Mentor}} \\
    \vspace{0.25cm}
    \begin{tabular}{r | l}
        Niklas Braun & niklas.braun@speedadmin.com \\
    \end{tabular}

    \vfill
    {\large \today}
\end{titlepage}

\tableofcontents
\newpage

% ===== CONTENT BEGINS HERE =====
\section{Project Background}

SpeedAdmin develops administrative software for music and art schools. The system manages entities such as students, teachers, lessons, and instruments, and automatically generates logs to record system events and changes.

These logs are essential for traceability and support, but they are highly detailed and often difficult to interpret. A single issue may involve many related log entries across different entities, making it challenging to gain a clear overview. As described in the project case, the logs are granular and verbose, and important information can easily be overlooked.

This creates a practical challenge for both support staff and customers. Understanding log data in a specific context currently requires manual analysis and technical knowledge, which can be time-consuming and inefficient. Therefore, there is a need for a solution that can transform raw log data into structured, understandable insights that better support real-world troubleshooting and analysis.


\section{Aim}
The aim of this project is to design and implement a containerised intelligent software system that enables natural language querying and automated analysis of SpeedAdmin's resource logs.

Users should be able to describe what they are trying to understand in plain language. Based on this input, the system will identify relevant log entries, generate concise summaries, explain them clearly, and highlight potential anomalies or points of interest.

The solution will follow a modular and model-agnostic architecture, ensuring that AI components adhere to defined input and output interfaces and can be replaced without major changes to the overall system. The project will demonstrate how intelligent software components can be integrated into a real-world enterprise environment, in line with the given objectives of the course.

\section{Objectives}
\textbf{Note:} Objectives marked with an asterisk ({\color{red}\textbf{*}}) are considered core objectives that we aim to achieve, while the others are additional features that we will implement if time allows.
\begin{itemize}
    \item AI-powered log summarization and anomaly detection based on user prompts{\color{red}\textbf{*}}
    \item Model-agnostic AI pipeline (swappable LLMs via Ollama){\color{red}\textbf{*}}
    \item Containerized microservice architecture (Docker){\color{red}\textbf{*}}
    \item REST API backend (Laravel){\color{red}\textbf{*}}
    \item React frontend for log viewing and prompting
    \item Optional: saving/sharing prompt results (MongoDB)
\end{itemize}

\section{Requirements Analysis}
\subsection{Functional Requirements}
\begin{itemize}
    \item \textit{Log Ingestion:} The system must be able to ingest and process log data from SpeedAdmin's existing logging infrastructure.
    \item \textit{Natural Language Querying:} Users should be able to query the logs using natural language prompts, and the system should return relevant summaries, insights, or anomaly detections based on the query.
    \item \textit{Model-Agnostic AI Pipeline:} The system should be designed to allow for easy swapping of underlying LLMs (e.g., GPT-4, LLaMA) via Ollama without requiring significant changes to the codebase.
    \item \textit{Containerization:} The entire system should be containerized using Docker to ensure consistency across different environments and ease of deployment.
    \item \textit{REST API Backend:} A RESTful API should be developed using Laravel to handle user queries, interact with the AI models, and manage data storage if necessary.
    \item \textit{Frontend Interface:} A user-friendly frontend should be developed using React to allow users to view logs, input queries, and receive responses from the AI system.
    \item \textit{Optional Data Persistence:} If implemented, the system should allow users to save or share their query results, which would require integration with a database such as MongoDB.
    \item \textit{Security and Access Control:} The system should implement appropriate security measures to protect sensitive log data and ensure that only authorized users can access the system, such as authentication.
\end{itemize}
\subsection{Non-Functional Requirements}
\begin{itemize}
    \item \textit{Performance:} The system should provide responses to user queries within an acceptable time frame (e.g., under 5 seconds) to ensure a smooth user experience.
    \item \textit{Scalability:} The system should be designed to handle increasing volumes of log data and user queries without significant degradation in performance.
    \item \textit{Usability:} The frontend interface should be intuitive and easy to navigate, allowing users of varying technical expertise to effectively utilize the system.
    \item \textit{Maintainability:} The codebase should be well-documented and structured to facilitate easy maintenance and future enhancements.
    \item \textit{Reliability:} The system should be robust and able to handle errors gracefully, ensuring that it remains operational even in the face of unexpected issues.
\end{itemize}

\section{Methods}
To achieve the project objectives, we have deviated from the original suggested tech stack by opting for Laravel instead of ASP.NET since our team has more experience with that framework. Making use of the following methods and tools:
\begin{itemize}
    \item \textit{Agile Development:}  We will follow an agile development approach, working in weekly iterative sprints along with regular team and mentor meetings.
    \item \textit{Version Control:} We will use Git for version control to manage our codebase and collaborate effectively as a team.
    \item \textit{GitHub:} We will host our code on GitHub, utilizing features such as issues, pull requests, and project boards to organize our work and track progress.
    \item \textit{Github Projects:} We will use a Github Project board to manage our tasks and milestones. This will allow us to track our progress, assign tasks and get our tasks done on deadline.
    \item \textit{Discord:} We will use Discord for team communication and coordination, ensuring that all team members are aligned and can easily share updates and discuss project-related matters.
\end{itemize}

\section{Architecture}
The system will be designed using a microservice architecture, with the following components and their respective technologies:
\begin{itemize}
    \item \textbf{Frontend} (\texttt{/frontend}): Web application that provides the user interface for viewing logs, inputting queries, and displaying AI-generated responses.
    \item \textbf{Backend} (\texttt{/backend}): Backend framework that serves as the REST API, handling user requests, processing queries, and interacting with the AI models and databases.
    \item \textbf{AI Service} (\texttt{/ai}): Separated service responsible for managing interactions with the LLMs, processing log data, and generating responses based on user queries. \\
          \textit{This service should not impact the backend's ability to function if the AI models are unavailable, and should be designed to handle such scenarios gracefully.}
    \item \textbf{Database} (\texttt{/database}): An SQL Server instance for storing any necessary relational data, and optionally an unstructured database instance for saving query results if that feature is implemented.
    \item \textbf{Containerization} (\texttt{/docker-compose.yml}): Docker will be used to containerize all components of the system, ensuring that they can be easily deployed and run in any environment.
\end{itemize}
\subsection{Tech Stack Overview}
\begin{itemize}
    \item \textit{AI Models:} We will leverage large language models (LLMs) such as GPT-4 or LLaMA for natural language processing tasks, including log summarization and anomaly detection.
    \item \textit{Ollama:} This tool will be used to manage and swap between different LLMs without requiring significant code changes.
    \item \textit{Laravel:} This PHP framework will be used to develop the REST API backend that handles user queries and interacts with the AI models.
    \item \textit{React:} The frontend will be built using React to create a responsive and user-friendly interface for viewing logs and inputting queries.
    \item \textit{MSSQL:} We will use Microsoft SQL Server for any necessary relational data storage, such as user information or log metadata.
    \item \textit{Docker:} We will use Docker to containerize the application, ensuring consistency across development, testing, and production environments.
    \item \textit{MongoDB (Optional):} If we choose to implement data persistence for saving or sharing query results, we will use MongoDB as our database solution.
\end{itemize}

\textbf{\textit{TODO: Add diagrams illustrating the architecture and interactions between components. (e.g., a high-level system architecture diagram, sequence diagrams for user interactions, deployment diagrams showing how the containers are orchestrated, etc.)}}
% TODO: Add diagrams illustrating the architecture and interactions between components. (e.g., a high-level system architecture diagram, sequence diagrams for user interactions, deployment diagrams showing how the containers are orchestrated, etc.)

\section{Risks and Mitigation Strategies}
\begin{itemize}
    \item \textbf{AI Model Performance:} The performance of the AI models may not meet expectations, leading to inaccurate or irrelevant responses.

          \textbf{Mitigation:} We will make use of prompt engineering techniques to optimize the performance of the AI models, and we will conduct thorough output testing and validation to ensure that the responses generated by the models are accurate and relevant to user queries.

    \item \textbf{Scalability Issues:} As the volume of log data and user queries increases, it may exceed the model's context window or lead to performance degradation.

          \textbf{Mitigation:} We will design the system with scalability in mind, implementing strategies such as log filtering, summarization and chunking to manage large volumes of data, and optimizing the AI pipeline to handle increased load efficiently.

    \item \textbf{Ollama performance on available hardware:} The performance of Ollama may be limited by the hardware resources available, which could impact the responsiveness of the AI service.

          \textbf{Mitigation:} We will test early and often to ensure that Ollama performs as expected on our available hardware, and we will resort to using quantized models if necessary to improve performance while still maintaining acceptable response quality.

          \textbf{Note:} \textit{The system should never use cloud-based LLMs to comply with GDPR regulations, so we will need to ensure that our local setup is sufficient for our needs.}

    \item \textbf{Team unfamiliarity with technologies:} Some team members may be unfamiliar with certain technologies or frameworks used in the project, which could lead to delays or challenges in development.

          \textbf{Mitigation:} We will allocate time for team members to learn and familiarize themselves with any new technologies, and we will encourage knowledge sharing and collaboration within the team to ensure that everyone is up to speed.
\end{itemize}

\section{Project Organization}
The project will be developed by a team of 6 members with the following roles and responsibilities:

\begin{itemize}
    \item \textbf{Project Manager} (Adrian): Oversees project coordination, timeline management, and stakeholder communication.

    \item \textbf{Backend Developers} (Adrian, Marks): Develop the REST API using Laravel, implement database interactions, and handle query processing and AI service integration.

    \item \textbf{Frontend Developers} (Dorina, Tomass, Jakub): Build the React application, design the user interface, and ensure an optimal user experience.

    \item \textbf{AI Specialists} (Balazs): Manage LLM interactions via Ollama, optimize prompts, and validate the accuracy and relevance of AI-generated responses.

    \item \textbf{DevOps Engineer} (team-wide responsibility): Containerize the application using Docker, configure development and production environments, and manage deployment processes.

\end{itemize}

We will use GitHub for version control and project management, utilizing branches with prefixes (e.g., \texttt{feature/}, \texttt{bugfix/}, \texttt{docs/}) to organize our work. We will also use GitHub Pull Requests for code reviews and merging, and we will maintain a project board to track tasks and progress. We will use Discord for team communication, scheduling regular meetings to discuss progress, address challenges, and plan next steps.

\textbf{Note:} \textit{While specific roles are assigned, all team members will collaborate and contribute across different areas of the project as needed to ensure its success.}

\section{Project Plan}
\hbadness=10000
\begin{table}[H]
    \centering
    \renewcommand{\arraystretch}{1.5}
    \begin{tabular}{|c|c|p{8cm}|c|}
        \hline
        \textbf{Week} & \textbf{Dates}  & \textbf{Focus}                                        & \textbf{Milestones} \\
        \hline
        8--9          & 19 Feb -- 5 Mar & Architecture design, repo setup, Docker skeleton      & {}                  \\
        \hline
        10            & 5 Mar           & Proposal submission                                   & \checkmark          \\
        \hline
        11--13        & 12--26 Mar      & Core backend API + DB connection, AI service skeleton &                     \\
        \hline
        14            & Easter Break    & \multicolumn{2}{c|}{\textit{No work planned}}                               \\
        \hline
        15            & 9 Apr           & Midterm seminar                                       & \checkmark          \\
        \hline
        16            & 16 Apr          & Midterm evaluation                                    & \checkmark          \\
        \hline
        17--19        & 23 Apr -- 7 May & Frontend, AI pipeline integration, prompt engineering &                     \\
        \hline
        20--21        & 14--21 May      & Testing, validation, report writing                   &                     \\
        \hline
        22            & 28 May          & Final presentation                                    & \checkmark          \\
        \hline
        23            & 2 Jun           & Code + report submission                              & \checkmark          \\
        \hline
    \end{tabular}
\end{table}


\section{Tentative Report Outline}
\begin{itemize}
    \item \textbf{Introduction:} Overview of the project, its background, and objectives.
    \item \textbf{Methodology:} Detailed description of the methods and technologies used in the project, including system architecture and design decisions.
    \item \textbf{Problem Analysis:} In-depth analysis of the problem being addressed, including any challenges or constraints encountered during development.
    \item \textbf{Requirements}: Comprehensive list of functional and non-functional requirements, along with any changes or updates made during the project.
    \item \textbf{Design}: Explanation of the system architecture, components, and interactions, supported by diagrams and illustrations.
    \item \textbf{Implementation:} Explanation of the development process, challenges faced, and how they were addressed.
    \item \textbf{Validation:} Discussion of how the results were validated, including testing methodologies and any limitations of the validation process.
    \item \textbf{Conclusion:} Summary of the project achievements and final thoughts.
\end{itemize}

\end{document}